Uma empresa de logística enfrenta o desafio de planear a entrega de encomendas para otimizar o uso da sua frota e dos seus motoristas.
O objetivo é maximizar o número de encomendas entregues, respeitando as restrições de capacidade dos veículos, disponibilidade de motoristas, tempos de trabalho e distâncias a percorrer.
Este problema é representativo das operações diárias enfrentadas por muitas empresas no setor de transportes e logística.

\subsection{Cenário}\label{subsec:cenario}
A empresa dispõe de uma frota de cinco veículos, cada um com características distintas de capacidade e velocidade média:
\begin{itemize}
    \item \textbf{Veículo A}: Capacidade de 5 m³, velocidade média de 50 km/h.
    \item \textbf{Veículo B} e \textbf{C}: Cada um com capacidade de 3 m³ e velocidade média de 70 km/h.
    \item \textbf{Veículo D} e \textbf{E}: Cada um com capacidade de 1 m³ e velocidade média de 80 km/h.
\end{itemize}
No início do planeamento, o armazém possui um inventário de 1000 encomendas, distribuídas em três categorias com diferentes volumes unitários:
\begin{itemize}
    \item \textbf{Tipo A}: 500 encomendas, volume de 0,017 m³ por caixa.
    \item \textbf{Tipo B}: 250 encomendas, volume de 0,053 m³ por caixa.
    \item \textbf{Tipo C}: 250 encomendas, volume de 0,026 m³ por caixa.
\end{itemize}
Além disso, a empresa tem à disposição oito motoristas, cada um com um turno diário de 8 horas.
Os veículos podem ser utilizados continuamente por diferentes motoristas ao longo das 24 horas do dia, desde que respeitadas as restrições de tempo de trabalho.
Cada entrega envolve um percurso de 15 km, e a capacidade máxima de cada veículo não pode ser excedida em momento algum.

\subsection{Objetivo}\label{subsec:objetivo}
O objetivo deste planeamento é determinar a alocação ideal de motoristas e encomendas aos veículos, para maximizar o número de encomendas entregues, respeitando as seguintes restrições:
\begin{enumerate}
    \item \textbf{Capacidade dos veículos}: O volume total das encomendas transportadas por cada veículo numa viagem não pode exceder a sua capacidade.
    \item \textbf{Tempo de entrega}: Cada veículo deve cumprir as entregas dentro do tempo disponível, considerando a distância e sua velocidade média.
    \item \textbf{Tempo máximo de uso dos veículos}: Cada veículo pode operar por, no máximo, 24 horas.
    \item \textbf{Turnos dos motoristas}: Cada motorista pode trabalhar no máximo 8 horas por dia.
    \item \textbf{Número de motoristas disponíveis}: O total de motoristas alocados não pode ultrapassar o limite disponível de 8 motoristas.
    \item \textbf{Limitação de encomendas}: O número total de encomendas entregues não pode ultrapassar o inventário disponível de cada tipo.
\end{enumerate}

\subsection{Desafio}\label{subsec:desafio}
A tarefa consiste em determinar:
\begin{enumerate}
    \item Quantas encomendas de cada tipo devem ser alocadas a cada veículo.
    \item Quantos motoristas devem ser alocados a cada veículo.
    \item O tempo de operação de cada veículo.
\end{enumerate}
Este problema será resolvido com a aplicação de métodos de programação linear, para maximizar o número total de encomendas entregues dentro das restrições impostas.