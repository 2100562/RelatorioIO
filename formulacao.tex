\subsection{Variáveis}\label{subsec:variaveis}
As variáveis utilizadas no problema foram definidas para representar as quantidades necessárias para a formulação do modelo de programação linear.
Estas variáveis são descritas a seguir:
\begin{itemize}
    \item \( x_{ij} \): Número de encomendas do tipo \( i \) alocadas ao veículo \( j \), onde:
    \begin{itemize}
        \item \( i \in \{1, 2, 3\} \): Representa os tipos de encomenda:
        \begin{itemize}
            \item \( i = 1 \): Encomendas do tipo A (\(0.017 \, \mathrm{m^3}\)).
            \item \( i = 2 \): Encomendas do tipo B (\(0.053 \, \mathrm{m^3}\)).
            \item \( i = 3 \): Encomendas do tipo C (\(0.026 \, \mathrm{m^3}\)).
        \end{itemize}
        \item \( j \in \{A, B, C, D, E\} \): Representa os veículos disponíveis:
        \begin{itemize}
            \item \( A \): Veículo com capacidade \( 5 \, \mathrm{m^3} \), velocidade \( 50 \, \mathrm{km/h} \).
            \item \( B, C \): Veículos com capacidade \( 3 \, \mathrm{m^3} \), velocidade \( 70 \, \mathrm{km/h} \).
            \item \( D, E \): Veículos com capacidade \( 1 \, \mathrm{m^3} \), velocidade \( 80 \, \mathrm{km/h} \).
        \end{itemize}
    \end{itemize}
    \item \( y_j \): Número de motoristas alocados ao veículo \( j \), onde \( j \in \{A, B, C, D, E\} \).
    \item \( t_j \): Tempo total de uso do veículo \( j \) (em horas), onde \( j \in \{A, B, C, D, E\} \).
\end{itemize}
Adicionalmente, foram utilizadas constantes para representar os limites físicos e operacionais do problema:
\begin{itemize}
    \item \( v_i \): Volume de uma encomenda do tipo \( i \) (em \( \mathrm{m^3} \)).
    \item \( C_j \): Capacidade do veículo \( j \) (em \( \mathrm{m^3} \)).
    \item \( v_j \): Velocidade média do veículo \( j \) (em \( \mathrm{km/h} \)).
    \item \( E_i \): Número total de encomendas disponíveis do tipo \( i \).
    \item \( T_{\max} \): Tempo máximo de uso diário de cada veículo (\( 24 \, \mathrm{horas} \)).
    \item \( D \): Distância necessária para cada entrega (\( 15 \, \mathrm{km} \)).
    \item \( H_{\text{turno}} \): Duração máxima do turno de cada motorista (\( 8 \, \mathrm{horas} \)).
    \item \( M_{\max} \): Número total de motoristas disponíveis (\( 8 \)).
\end{itemize}

\subsection{Função Objetivo}\label{subsec:funcobj}
A formulação matemática do problema foi realizada para maximizar o número total de encomendas entregues, respeitando as restrições operacionais e de capacidade impostas.
A função objetivo é definida como:
\[
    \text{Maximizar } Z = \sum_{i=1}^{3} \sum_{j \in \{A, B, C, D, E\}} x_{ij}
\]
Onde:
\begin{itemize}
    \item \( x_{ij} \): Número de encomendas do tipo \( i \) alocadas ao veículo \( j \).
    \item \( i \in \{1, 2, 3\} \): Tipos de encomendas.
    \item \( j \in \{A, B, C, D, E\} \): Veículos disponíveis.
\end{itemize}

\subsection{Restrições}\label{subsec:restricoes}
O modelo está sujeito às seguintes restrições:
\begin{enumerate}
    \item \textbf{Capacidade dos veículos}: O volume total das encomendas transportadas por cada veículo não pode exceder a sua capacidade:
    \[
        \sum_{i=1}^{3} v_i \cdot x_{ij} \leq C_j, \quad \forall j \in \{A, B, C, D, E\}
    \]
    Onde:
    \begin{itemize}
        \item \( v_i \): Volume de uma encomenda do tipo \( i \) (em \( \mathrm{m^3} \)).
        \item \( C_j \): Capacidade do veículo \( j \) (em \( \mathrm{m^3} \)).
    \end{itemize}
    \item \textbf{Tempo necessário para entrega}: O tempo total para as entregas de cada veículo não pode exceder o tempo disponível:
    \[
        \sum_{i=1}^{3} \frac{x_{ij} \cdot D}{v_j} \leq t_j, \quad \forall j \in \{A, B, C, D, E\}
    \]
    Onde:
    \begin{itemize}
        \item \( D \): Distância necessária para cada entrega (\( 15 \, \mathrm{km} \)).
        \item \( v_j \): Velocidade média do veículo \( j \) (\( \mathrm{km/h} \)).
        \item \( t_j \): Tempo total de uso do veículo \( j \) (em horas).
    \end{itemize}
    \item \textbf{Tempo máximo por veículo}: O tempo de uso de cada veículo não pode exceder 24 horas:
    \[
        t_j \leq 24, \quad \forall j \in \{A, B, C, D, E\}
    \]
    \item \textbf{Turnos dos motoristas}: O tempo de uso de cada veículo deve estar limitado ao número de motoristas alocados multiplicado pelo turno máximo de trabalho:
    \[
        t_j \leq y_j \cdot H_{\text{turno}}, \quad \forall j \in \{A, B, C, D, E\}
    \]
    Onde:
    \begin{itemize}
        \item \( y_j \): Número de motoristas alocados ao veículo \( j \).
        \item \( H_{\text{turno}} \): Duração máxima do turno de cada motorista (\( 8 \, \mathrm{horas} \)).
    \end{itemize}
    \item \textbf{Número total de motoristas}: O total de motoristas alocados não pode exceder o número disponível:
    \[
        \sum_{j \in \{A, B, C, D, E\}} y_j \leq M_{\text{max}}
    \]
    Onde \( M_{\text{max}} = 8 \) motoristas.
    \item \textbf{Limitação de encomendas disponíveis}: O número total de encomendas entregues por tipo não pode exceder o inventário disponível:
    \[
        \sum_{j \in \{A, B, C, D, E\}} x_{ij} \leq E_i, \quad \forall i \in \{1, 2, 3\}
    \]
    Onde \( E_i \): Total de encomendas disponíveis do tipo \( i \).
    \item \textbf{Restrições de não negatividade}: Todas as variáveis devem ser não negativas:
    \[
        x_{ij} \geq 0, \quad y_j \geq 0, \quad t_j \geq 0, \quad \forall i, j
    \]
\end{enumerate}

\subsection{Formulação Final}\label{subsec:formulacao-final}
Com base na função objetivo e nas restrições descritas anteriormente, a formulação final do problema de programação linear é apresentada a seguir:
\[
    \text{Maximizar } Z = \sum_{i=1}^{3} \sum_{j \in \{A, B, C, D, E\}} x_{ij}
\]
s.a:
\begin{enumerate}
    \item \[
              \sum_{i=1}^{3} v_i \cdot x_{ij} \leq C_j, \quad \forall j \in \{A, B, C, D, E\}
    \]
    \item \[
              \sum_{i=1}^{3} \frac{x_{ij} \cdot D}{v_j} \leq t_j, \quad \forall j \in \{A, B, C, D, E\}
    \]
    \item \[
              t_j \leq 24, \quad \forall j \in \{A, B, C, D, E\}
    \]
    \item \[
              t_j \leq y_j \cdot H_{\text{turno}}, \quad \forall j \in \{A, B, C, D, E\}
    \]
    \item \[
              \sum_{j \in \{A, B, C, D, E\}} y_j \leq M_{\max}
    \]
    \item \[
              \sum_{j \in \{A, B, C, D, E\}} x_{ij} \leq E_i, \quad \forall i \in \{1, 2, 3\}
    \]
\end{enumerate}
\[
    x_{ij} \geq 0, \quad y_j \geq 0, \quad t_j \geq 0, \quad \forall i, j
\]