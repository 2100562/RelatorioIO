O presente relatório visa apresentar o desenvolvimento e a resolução de um problema prático de programação linear, uma área central na investigação operacional.
Este trabalho foi realizado por um grupo composto por três estudantes, cada um assumindo funções específicas para garantir uma abordagem colaborativa e eficiente.
Desde o início, foram estabelecidas as seguintes responsabilidades: João Valadares como líder, Pedro Morais como \textit{designer}, e Hugo Gonçalves como analista.
Esta organização permitiu estruturar o trabalho de forma eficaz, desde a definição do problema até à obtenção dos resultados.

A temática escolhida, o planeamento de entregas de uma empresa de logística, foi selecionada devido à sua relevância prática e impacto no dia a dia.
O problema consiste em maximizar o número de encomendas entregues, utilizando uma frota de veículos com capacidades e velocidades variadas, bem como um número limitado de motoristas e restrições de tempo.
Este cenário reflete os desafios reais enfrentados por empresas logísticas, como a necessidade de otimizar recursos para atender à demanda de forma eficiente.

Para a realização deste trabalho, foram utilizadas várias ferramentas tecnológicas, incluindo o Moodle para partilha de informações, o Microsoft Teams para reuniões e comunicação, o LaTeX para a elaboração do relatório, e o Google Colab para a implementação da programação e resolução do problema.
Estas ferramentas facilitaram a colaboração em equipa e a análise sistemática dos dados.

Este relatório detalha todas as etapas do processo, desde a formulação do problema, definição das variáveis, função objetivo e restrições, até à apresentação dos resultados obtidos e conclusões.
Pretende-se não apenas encontrar a solução ótima para o problema proposto, mas também explorar cenários alternativos que possam oferecer \textit{insights} adicionais sobre a gestão de recursos e o impacto das condições impostas.